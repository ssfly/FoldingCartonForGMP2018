\section{Overview}\label{sec:overview}


Figure~\ref{fig:overview} shows the overview of our algorithm. 
As shown in Figure~\ref{fig:overview}(a), the input 2D design layout of a carton consists of a set of cutting edges (solid lines) and folding edges (dashed lines).
%
Given the 2D layout, an undirected graph is built and a set of polygonal panels are extracted by finding the minimum cycles in the graph, as Figure~\ref{fig:overview}(b) shows. 
%Each face is filled with a unique color. %by ignoring the holes in the plane
The 2D layout therefore can be represented by a polymesh $\mathcal{L}=(V,E,P)$, where $V$ is the set of vertexes, $E$ is the set of all edges, and $P$ is the set of panels. 
The edge set $E=E_c\cup E_f$, where $E_c$ is the set of cutting edges, and $E_f$ is the set of folding edges.
%
To build a 3D carton model $\mathcal{M}=(V, E, P)$ from the 2D layout $L$, while they share the same topology, we compute the 3D coordinates of all vertexes in $V$. 
%

However, it is not intuitive to analytically define the desired final 3D shape from a 2D layout. 
One possible way is detecting all the possible geometric constraints such as vertex merging, panel parallelism, orthogonality between adjacent panels, and then integrating all these possible constraints together to form a large equation system. 
But the challenge is that these local constraints can not well describe complicated and creative designs. 
Moreover, there are many ambiguities when detecting these constraints in a 2D layout that consists of many edges and panels in the same shape. 
%
Based on the observation that humans usually fold edges with a right angle to get a rough shape and then merge vertexes or edges to get a stable 3D carton, we propose a two-step algorithm. 
% Our algorithm consists of two steps. 
First, an initial 3D model (Figure~\ref{fig:overview}(c)) is constructed based on a specific angle along each folding edge, as described in Sec.~\ref{sec:initialization}.
The user can then manipulate and explore 3D shapes of the canton based on a series of suggestive operations provided by our system. 
%
The final model is shown in Figure~\ref{fig:overview}(e).
Details of each step will be introduced as following. 

% is finally built through the optimization based on the information acquired from user interaction. 
%\cxj{Modify this overview when you have new figure.}
%{\color{blue}{Furthermore, our system allows users modify the final 3D model to the desired shape, and automatically enforce the shape constraints reversely to the flat polymesh to get a deformable layout as shown in Figure~\ref{fig:overview} (f).}}


\comments{
  a flat polymesh $L$ is created from a 2D design layout of a box, then we deform the input polymesh into its 3D realization $R_i$ according to the predicted angles along each of its fold edges, and through optimization, generate final model $R_f$. A polymesh consists of a set of vertices, edges and faces $M = (V,E,F)$, the number of vertexes $V$, edges $E$ and faces $F $ vary from one mesh to another. However, a pair of $(L,R_f)$ as the 2D layout and its corresponding 3D realization share the same topology and therefore they have the same number of vertices, edges and faces. A flat mesh as a 2D layout $L$ has its $z$ component of each vertex set to be a constant zero: $X_z(\mathbf{v}) \equiv 0$ where $X = (X_x,X_y,X_z)$ is the vertex coordinate, and its normal of each face set as $(0,0,1)^T$: $\mathbf{n}(f) \equiv (0,0,1)^T$, where $f \in F$.
}


