\section{Related Work}\label{sec:relatedwork}
\subsection{Folding problem}
Folding two-dimensional pattern into three-dimensional objects has been studied in various fields, such as rigid origami folding, curved folding and carton folding. Despite the 2D pattern as an input, all of these problems need extra 3D information or interactions, while this paper chooses the 2D layout as only input.

Rigid origami folding is similar to our problem, while the folding motion is mostly computed based on the given angle of creases, and consider the geometry of origami in kinetic motion~\cite{tachi2009simulation,tachigeometric}. The thickness problem of origami is also a direction of origami related research these years, and its goal is to simulate the folding behaviour exactly the same to when folding paper with zero thickness by designing different connecting structures between panels~\cite{chen2015origami,2016arXiv160105747K,tachi2011rigid}. Thick origami folding problem also has a target configuration to approximate.

Curved folding from a single sheet of paper is a variation of computational origami which considers curved lines as a part of creases, and can be treated as a practical instance of developable surface. Kilian et al.~\cite{Kilian:2008:CF:1360612.1360674} presented an optimization based framework to approximate the given geometric data. Solomon et al.~\cite{Solomon:2012:FDS:2346796.2346817} provided a subdivision based modeling scheme involving curved paper structure with folding angles on creases as input. Kilian et al.~\cite{Kilian:2017:SAC:3087678.3015460} studied the deformation of curved folded surfaces after the folding motion actuated by pulling a network of string. The extra 3D information is represented as target 3D objects, folding angles or external forces above.

There are also some methods to solve related problems of carton folding. 
Song et al.~\cite{Song:2000:MPA:892954} modeled foldable objects as tree like multilink objects and used PRMs~(probabilistic roadmap methods~\cite{Kavraki:1994:PRP:891758}) to find a sequence of motions to transform one configuration of a foldable object into another configuration. 
Mullineux et al.~\cite{Mullineux:2010:CSC:1739328.1739673} provided a simulation framework of the carton during erection using a constraint-based approach. Both these work required the target state as a premise, while our work aims to generate the target configuration.

Folding 2D layouts with interactive tools visualizes the folding behaviour of a single piece of paper. Thiel~\cite{Thiel1998} provided a virtual origami system including the user interface to model folded paper and show animations of folding process. Kishi et al.~\cite{Kishi:1998:OFP:786112.786279} allowed users to create and edit the origami properties over the Web. Nimnual et al.~\cite{Nimnual2007Virtual} presented an application for package folding practices in a virtual space. Although these applications can model folded paper well, they need given parameters to construct the model, or their final goal is flat folding which can also be seen as a premier to the problem.

In addition, the above works cannot edit corresponding 3D model and optimize its layout in turn, which is one of the advantages of our system.

%\subsection{Paper folding problem}
%Various types of paper crafts have been studied in the field of computation and mathematics. Origami is the Japanese traditional paper art of making different kind of objects by a single sheet of paper, and has been long studied hundreds of years ago~\cite{KANADE1980279}. \reply{The simulation of rigid origami is similar to our carton folding problem, while the folding motion is mostly computed based on the given angle of creases, and consider the geometry of origami in kinetic motion~\cite{tachi2009simulation,tachigeometric}. In this paper, cartons are folded into 3D model without knowing the prior of crease angles, and can be deformed through optimization while origami folding cannot deal with the deformation. The thickness problem of origami is also a direction of origami related research these years~\cite{chen2015origami,2016arXiv160105747K,tachi2011rigid}, while thickness is not considered in our problem.}
%
%\reply{Curved folding from a single sheet of paper is a variation of computational origami which considers curved lines as a part of creases, and can be treated as a practical instance of developable surface. Kilian et al.~\cite{Kilian:2008:CF:1360612.1360674} presented an optimization based framework to approximate the given geometric data. Solomon et al.~\cite{Solomon:2012:FDS:2346796.2346817} provided a subdivision based modeling scheme involving curved paper structure with the folding angle on creases as input. Kilian et al.~\cite{Kilian:2017:SAC:3087678.3015460} studied the deformation of curved folded surfaces after the folding motion actuated by pulling a network of string. Despite the 2D expanded layout as an input, curved folding problem always has extra information, such as approximate shape, folding angles or external force, while the carton design layout is our only input.}
%
%While researching on origami in the field of computational algorithm and geometric analysis, the simulation system is developed to visualize the folding behaviour of a single piece of paper. Thiel~\cite{Thiel1998} provides a virtual origami system including the user interface to model folded paper and show animations of folding process. Kishi et al.~\cite{Kishi:1998:OFP:786112.786279} allowed users to create and edit the origami properties over the Web. Nimnual et al.~\cite{Nimnual2007Virtual} presented an application for package folding practices in a virtual space. Although these applications can model folded paper well, they need given parameters to construct the model. How to align constraints to structural designs by implementing shape optimization is our main concern.
%
%There are also some methods to solve related problems of carton folding. 
%Song et al.~\cite{Song:2000:MPA:892954} modeled foldable objects as tree like multilink objects and used PRMs~(probabilistic roadmap methods~\cite{Kavraki:1994:PRP:891758}) to find a sequence of motions to transform one configuration of a foldable object into another configuration. 
%Mullineux et al.~\cite{Mullineux:2010:CSC:1739328.1739673} provided a simulation framework of the carton during erection using a constraint-based approach. Both these work required the target state as a premise, while our work aim to generate the target configuration.
%
%The complexity of folding to polyhedra problem has also been studied for decades, Lubiw~\cite{Lubiw1996When} provided an dynamic programming algorithm based on Aleksandrov's theorem to test whether a polygon can be folded into polyhedra which takes $O(n^2)$ time and space. 
%O'Rourke~\cite{O'Rourke:1998:FUC:646319.686376} examined three open problems on the subject of folding and unfolding. 
%Biedl et al.~\cite{Biedl2004When} studied in polynomial time to solve the question of when is the graph orthogonally convex polyhedra given a graph, edge length and facial angles, also shown that it's NP-hard to decide whether the graph is orthogonally polyhedra or not. Rather than the given graph, Biedl et al.~\cite{Biedl:2005:NFP:1090462.1646553} proved that if given a net along with the dihedral angle at each crease, we can know whether a net can be folded to a polyhedron in polynomial time, but it becomes NP-hard without the angles even adding constraints on orthogonal polyhedron, which results in more difficulties on more complex input.
%Compared to our desired result, a polyhedron is a set of polygons without overlap, nevertheless, our 3D model contains small faces that needs to be fixed to another panel. 
%These works above justify our problem being hard to solve caused by even more intricate inputs.

\subsection{3D reconstruction of line drawings}
\reply{Although having the same goal of constructing 3D models from 2D layouts including vertexes and edges, there is a one-to-one correspondence between the vertexes in the 2D line drawing and 3D model, while our problem recovers many-to-one correspondences and the 3D construction process is different.} 

A line drawing is defined as a 2D projection of a object containing its vertexes and edges. 3D interpretation of line drawings have been studied for a long time. 
%The main problem is in object reconstruction given its projection on two-dimensional planes. 
Some researchers treat this task as an optimization problem. 
Marill~\cite{Marill:1991:EHI:113057.113061} first proposed the principle of minimizing the standard deviation (MSDA) of angles to emulate the interpretation of line drawings as 3D objects. 
%
This new criterion was used by many other researchers soon after. 
Leclerc et al.~\cite{Leclerc1992An} combined MSDA with the deviation from planarity as objective terms. 
Cao et al.~\cite{Cao:2005:ORS:1097114.1097658} added a symmetry metric of 3D objects to get more complicated results. 
Some other researchers tried to solve this problem from the point of view of information theory.
%
Marill~\cite{Marill1992Why} minimized the description length of objects based on the idea that humans usually pick the simplest one from infinite possibilities when they see the line drawing. 
Shoji et al.~\cite{Shoji20013} minimized the entropy of the angle distribution between line segments by a genetic algorithm. 
Later, the strategy of splitting and merging was used to reconstruct line drawings of complex 3D objects~\cite{10.1109/TPAMI.2010.49,10.1109/CVPR.2014.94}.   
		 
Different from line drawings, our input is the expanded structural layouts in 2D planes of 3D carton objects.
The 3D carton model is constructed based on the folding procedure instead of the mathematical projection from 3D to 2D.

\subsection{Geometrical shape optimization}
%\reply{The basic idea of our paper is  shape optimization under a set of simple constraints.}
In geometrical shape processing, many algorithms have been proposed to enforce shape constraints, and have been successfully used in interactive tools and physical simulation~\cite{Botsch:2006:PCP:1281957.1281959,Igarashi:2005:ASM:1186822.1073323}. 
Bouaziz et al.~\cite{Bouaziz:2012:SSD:2346796.2346802} unified a large variety of geometric constraints into one optimization framework, and provided a simple and robust implementation. 
Poranne et al.~\cite{Poranne2013Interactive} described an interactive method to manipulate and optimize polyhedral meshes under constraints with a linear-time algorithm. 
%
Tang et al.~\cite{Tang:2014:FPM:2601097.2601213} solved constrained equations by Newton-type method in a fast way, and provided an interactive system to model meshes constrained by equalities and inequalities. 
Deng et al.~\cite{Deng2015} developed an interactive tool to explore architectural design with shape constraints, and provided an optimization method to enforce hard constraints and soft constraints at the same time. 

\xjmd{The studies mentioned above focus on optimizing existing 3D shapes with geometric constraints, and propose different optimization solutions.
%
2D layouts are not involved in their problems.
In contrast, our goal is to recover the structure and topology from an expanded layout to form a 3D carton model.
The geometric constraints are automatically detected and presented for users to explore during the folding process. }
%correspondence and our constraints do not need satisfy properties like fairness. 
%In our paper, we implement the shape optimization by the method introduced in \cite{Bouaziz:2012:SSD:2346796.2346802} for its robustness and simplicity.



