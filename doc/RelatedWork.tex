\section{Related Work}\label{sec:relatedwork}
\subsection{Paper folding problem}
Various types of paper crafts have been studied in the field of computation and mathematics. Origami is the Japanese traditional paper art of making different kind of objects by a single sheet of paper, and has been long studied since 1970s~\cite{KANADE1980279}.

While researching on origami in the field of computational algorithm and geometric analysis, the simulation system is developed to visualize the folding behaviour of a single piece of paper. Thiel provides a virtual origami system including the user interface to model folded paper and show animations of folding process~\cite{Thiel1998}. Kishi et al. allows users create and edit the origami properties over the Web~\cite{Kishi:1998:OFP:786112.786279}. Nimnual et al. presented an application for package folding practices in a virtual space~\cite{Nimnual2007Virtual}. Although these applications can model folded paper well, they need given parameters to construct the model, and how to align constrains to structural layout by implementing shape optimization is our main concern.

There are also some methods to solve the problem of carton folding. 
Song et al. modeled foldable objects as tree like multilink objects and used PRMs~(probabilistic roadmap methods~\cite{Kavraki:1994:PRP:891758}) to find a sequence of motions to transform some configuration of a foldable object into another configuration~\cite{Song:2000:MPA:892954}. 
Mullineux et al. provided a simulation of the carton during erection using a constraint-based approach. Both these work required the target state as a premise, while our work aim to generate the target configuration~\cite{Mullineux:2010:CSC:1739328.1739673}.

The complexity of folding to polyhedron problem has also been studied for decades, Lubiw provided an dynamic programming algorithm based on Aleksandrov's theorem to test whether a polygon can be folded into polyhedra which takes $O(n^2)$ time and space~\cite{Lubiw1996When}. 
O'Rourke examined three open problems on the subject of folding and unfolding~\cite{O'Rourke:1998:FUC:646319.686376}. 
Biedl et al. has studied in polynomial time to solve the question of when is the graph orthogonally convex polyhedra given a graph, edge length and facial angles, also shown that it's NP-hard to decide whether the graph is orthogonally polyhedra or not~\cite{Biedl2004When}. Rather than the given graph, Biedl et al. proved that if given a net along with the dihedral angle at each crease, we can know whether a net can be folded to a polyhedron in polynomial time, but it becomes NP-hard without the angles even adding constrains on orthogonal polyhedron, which results in more difficulties on more complex input~\cite{Biedl:2005:NFP:1090462.1646553}.
Compared to our desired result, polyhedron is a set of polygons without overlap, nevertheless, our 3D model contains paste faces that needs to be fixed to another panel. 
These works above justify our problem being hard to solve caused by even more intricate inputs.

\subsection{Reconstruction from single line drawings} 
A line drawing is defined as a 2D projection of object containing its vertexes and edges. Line drawings of three-dimensional objects have long been studied, and the main problem is still in object reconstruction given its projection on two-dimensional planes. 
Some researchers treat this task as optimization problem. 
Marill proposed MSDA~(Minimize the Standard Deviation of Angles) principle to emulate the interpretation of line drawings as 3D objects~\cite{Marill:1991:EHI:113057.113061}. 
This new criterion is used by many other researchers later. 
Leclerc et al. combined MSDA with the deviation from planarity as objective terms~\cite{Leclerc1992An}. 
Cao et al. added symmetry measure of the objects to get more complicated results~\cite{Cao:2005:ORS:1097114.1097658}. 
Some other researchers try to solve this problem from the information theoretic point of view. 
Marill minimized the description length of objects based on the idea that we usually pick the simplest one from infinite possibilities when we see the line drawing~\cite{Marill1992Why}. 
Shoji et al. implemented the principle of minimizing the entropy of angle distribution between line segments using genetic algorithm~\cite{Shoji20013}. 
%
Different from the input above, ours are expanded structural layout of three-dimensional objects in 2D planes, the final model is constructed based on the folding process instead of mathematical presentation.

\subsection{Shape optimization}
Multiple algorithms have been proposed to enforce shape constrains, and have been used successfully for interactive tools and physical simulation~\cite{Botsch:2006:PCP:1281957.1281959,Igarashi:2005:ASM:1186822.1073323}. 
Bouaziz et al. unified a large variety of geometric constrain into one optimization framework, and provided simple, and robust implementation~\cite{Bouaziz:2012:SSD:2346796.2346802}. 
Poranne et al. described an interactive method to manipulate and optimize polyhedral meshed with constrains with a linear-time algorithm~\cite{Poranne2013Interactive}. 
Tang et al. solved constrained equations by Newton-type method in a fast way, and provided an interactive system to model meshed constrained by equalities and inequalities~\cite{Tang:2014:FPM:2601097.2601213}. 
Deng et al. developed an interactive tool to explore architectural design with shape constrains, and provided an optimization method to enforce hard constraints and soft constraints at the same time~\cite{Deng2015}. 
\cxj{In comparison, we provide....}
{\color{blue}{In our paper, we implement the shape optimization by the method introduced in \cite{Bouaziz:2012:SSD:2346796.2346802} for its robustness and simplicity.}}
%In our paper, we implement the shape optimization by the method introduced in \cite{Bouaziz:2012:SSD:2346796.2346802} for its robustness and simplicity.



