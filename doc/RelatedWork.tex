\section{Related Work}\label{sec:relatedwork}
\subsection{Folding problem}

Origami folding, which produces a 3D shape from a single planar sheet, has been widely studied for decades. Earlier techniques focus on simulating this folding process with a 2D crease pattern and the corresponding configuration of fold angles~\cite{Thiel1998,Kishi:1998:OFP:786112.786279,Nimnual2007Virtual}. Typically, these simulation systems require the folding order and the angles of the creases to exist within a 2D pattern. 
%
If a 2D crease pattern is provided for a mountain or valley design, a continuous process can be simulated to design transformable and deployable structures~\cite{tachi2009simulation,tachigeometric}.
%
The above simulation systems can be applied primarily for paper origami with zero-thickness. Recently, many mechanical engineering techniques~\cite{tachi2011rigid,chen2015origami,2016arXiv160105747K} have been proposed to develop the kinetic synthesis of thick panels that can be folded identically like zero-thickness origami.
%
These simulation systems are limited to flat-foldable rigid origami with known mountain or valley crease designs. In comparison, our system only takes the 2D crease pattern as input without knowing the folding angles of the creases. 

 
There are also some methods for solving related problems in carton folding. 
Song et al.~\cite{Song:2000:MPA:892954} modeled foldable objects as tree-like multilink objects and used probabilistic roadmap methods~\cite{Kavraki:1994:PRP:891758} to find a sequence of motions to transform one configuration of a foldable object into another configuration. 
Mullineux et al.~\cite{Mullineux:2010:CSC:1739328.1739673} provided a simulation framework for carton construction by considering geometric constraints for assembly.
Both of these two methods require the target 3D state to be given as a premise, while our work aims to recover the 3D configuration of the folded carton from a 2D layout.

 
Compared with the straight creases in the above approaches, curved folding, which involves the use of curved folds in the crease pattern, has also drawn a lot of attention in computer graphics.  %
 Kilian et al.~\cite{Kilian:2008:CF:1360612.1360674} presented an optimization-based framework to reconstruct a 2D development from a reference 3D surface.
 %
 Solomon et al.~\cite{Solomon:2012:FDS:2346796.2346817} introduced a discrete paradigm to model developable surfaces based on 2D configurations, where the crease pattern and the corresponding crease angles are pre-defined.
 %
Recently, Kilian et al.~\cite{Kilian:2017:SAC:3087678.3015460} designed practical mechanisms to fabricate a curved folded surfaces simply by pulling a network of strings. 
%the deformation of curved folded surfaces after the folding motion actuated by pulling a network of string. 
%
By contrast, we focus on an inverse problem of recovering the 3D shape from a 2D planar pattern without knowing any 3D information.





\subsection{3D reconstruction of line drawings}
Although the aforementioned methods have the same goal of constructing 3D models from 2D layouts, including vertexes and edges, there is a one-to-one correspondence between the vertexes in the 2D line drawing and 3D model; however, our problem recovers many-to-one correspondences and the 3D construction process is different.

A line drawing is defined as a 2D projection of an object that contains its vertexes and edges. 3D interpretations of line drawings have been studied for a long time. 
%The main problem is in object reconstruction given its projection on two-dimensional planes. 
Some researchers treat this task as an optimization problem. 
Marill~\cite{Marill:1991:EHI:113057.113061} first proposed the principle of minimizing the standard deviation (MSDA) of angles to emulate the interpretation of line drawings as 3D objects. 
%
This new criterion was used by many other researchers soon after. 
Leclerc et al.~\cite{Leclerc1992An} combined MSDA with the deviation from planarity as objective terms. 
Cao et al.~\cite{Cao:2005:ORS:1097114.1097658} added a symmetry metric of 3D objects to obtain more complicated results. 
Other researchers tried to solve this problem from the point of view of information theory.
%
Marill~\cite{Marill1992Why} minimized the description length of objects based on the idea that humans usually pick the simplest option from infinite possibilities when they see a line drawing. 
Shoji et al.~\cite{Shoji20013} minimized the entropy of the angle distribution between line segments by utilizing a genetic algorithm. 
Later, the strategy of splitting and merging was used to reconstruct the line drawings of complex 3D objects~\cite{10.1109/TPAMI.2010.49,10.1109/CVPR.2014.94}.   
		 
Different from line drawings, our input consists of the expanded structural layouts of 3D carton objects as displayed in 2D planes.
%
The 3D carton model is constructed based on the folding procedure instead of the mathematical projection from 3D to 2D.

\subsection{Geometrical shape optimization}
%\reply{The basic idea of our paper is  shape optimization under a set of simple constraints.}
In geometrical shape processing, many algorithms have been proposed to enforce shape constraints, which have been successfully used in interactive tools and physical simulation~\cite{Botsch:2006:PCP:1281957.1281959,Igarashi:2005:ASM:1186822.1073323}. 
Bouaziz et al.~\cite{Bouaziz:2012:SSD:2346796.2346802} unified a large variety of geometric constraints into one optimization framework, thus providing a simple and robust implementation. 
Poranne et al.~\cite{Poranne2013Interactive} described an interactive method to manipulate and optimize polyhedral meshes under constraints with a linear-time algorithm. 
%
Tang et al.~\cite{Tang:2014:FPM:2601097.2601213} solved constrained equations by employing a Newton-type method in a fast way, which yielded an interactive system to model meshes constrained by equalities and inequalities. 
Deng et al.~\cite{Deng2015} developed an interactive tool to explore architectural design with shape constraints.
They also provided an optimization method to enforce hard constraints and soft constraints at the same time. 

The studies mentioned above focus on optimizing existing 3D shapes with geometric constraints, by proposing different optimization solutions.
%
2D layouts are not involved in their problems.
In contrast, our goal is to recover the structure and topology from an expanded layout to form a 3D carton model.
The geometric constraints are automatically detected and presented for users to explore during the folding process.  
%correspondence and our constraints do not need satisfy properties like fairness. 
%In our paper, we implement the shape optimization by the method introduced in \cite{Bouaziz:2012:SSD:2346796.2346802} for its robustness and simplicity.



