\section{Algorithm}\label{sec:optimization}

\comments{
Assuming every face in the carton model is rigid, which means the interior angle in each plane stays the same.  
%
Furthermore, planes are connected by hinges at the boundary of patches, and the planar layout has its front and back.}

In this section, we explain our algorithm in detail. 
Initially, assume that the 2D layout $L$ lies on the $XOZ$ plane, so that each vertex $\mathbf{v}_i=(x_i,y_i,z_i)$ in $V$ has $z_i=0$. 
Each face in $F$ has a normal $\mathbf{n}_i=(0,0,1)^T$.
%
In the first step of our algorithm, our system automatically computes the normal of the desired 3D carton based simple rules of orthogonality and parallelism.
In the second step, the carton shape is refined based on a set of possible geometric constraints.


\subsection{Shape Initialization}\label{sec:initialization}

\cxj{re-write this section. Describe clearly about the parameters, the objective, and methods. }

In this section we explain the reason why using a specific angle to each fold edge, and constructing our initialized model. The basic ideal is to interpret the folded state of a box as a series of rotation angles along each edge, and by setting specific value of angles which is $\pi/2$, we can have a rough model to assist the later optimization.
		
Observed from existing data in the Internet, most of the traditional cartons are cuboid, as the examples shown in Figure~\ref{fig:realdata} (a) and (b), they are used to put in files or delivering daily supplies. Although there is a recent trend to design more complicated layouts to attract consumers Figure~\ref{fig:realdata} (c), the shape of these unusual cartons is similar to orthogonality boxes as their functions are still packaging commodity. Based on this observation, we set $\pi/2$ as the value of rotation angle to each fold edge, and have the initialized model as Figure~\ref{fig:initial}.

As shown in Figure~\ref{fig:initial} (a), the traditional cartons as cuboid boxes can reach an ideal state, while the novel designs need a little refinement like the four examples in Figure~\ref{fig:initial} (b). Take the hexagonal box as an example, users only need to assign six paste faces into corresponding surface, the model of a feasible carton will be generated. As a result, we interactively allow users to add these constrains into our system and optimize to a desired model.

\begin{figure}
	\centering
	\includegraphics[width=0.9\textwidth]{images/realdata.jpg}
	\caption{Two traditional cartons (a), (b) and one unusual carton (c) with their corresponding layouts.}
	\label{fig:realdata}
\end{figure}

\begin{figure}
	\centering
	\includegraphics[width=0.9\textwidth]{images/initial.jpg}
	\caption{Eight different initialization results. Four initial cartons shown in the second column of (a) can reach the ideal state, the other four cartons need further refine (b).}
	\label{fig:initial}
\end{figure}

%%%%%%%%%%%%%%%%%%%%%%%%%%%%%%%%%%%%%%%%%%%%%%%%%%%%%%%%%%%%%%%%%%%%%


\subsection{Shape Refinement}

%there is still a need to refine the results that have not folded into pleasing results. 
Though not perfect, the initial 3D shape generated by the simple rule provides a good start point for generating the final carton model. 
%
To further refine the 3D shape, the 3D coordinates of all the vertexes are then computed based on a set of shape constraints.
%
%the main idea is to prescribe the shape constrains by a set of vertexes of the polymesh. 
Moreover, with the extra information acquired from user interaction, we can finally construct the desired 3D realization.
In this step, the coordinate of vertexes are chosen as our objective instead of angles on folding edges.
This is mainly because that the geometric constrains in the 3D shape can be more simply and intuitively represented by 3D vertexes.
% and we can implement the algorithm introduced by Bouaziz et al. easily~\cite{Bouaziz:2012:SSD:2346796.2346802}.

\cxj{describe possible suggestions on the geometric constraints. I moved the detection part here. } 

%\subsection{Aided Detection}

%While the initial 3D model with simple angle folding provides a rough idea about the carton shape, 
Our system  automatically detect a group of possible shape constraints to form the final 3D model, such as vertex merging, shape symmetry, and so on. 
%
These guiding operations are provided to the user at bottom of the user interface, for the user to quickly select and explore the carton shapes.
In our system, \cxj{two} types of shape refinement operations, vertex merging, and shape symmetry, are automatically detected.

%
%In order to assist users construct the final carton interactively, we also provide symmetry detection and merging points detection to improve the efficiency of constructing models .

\noindent
\textbf{Vertex merging} is detected among the nearby vertexes. 
For irregular shapes consists of non-rectangular faces or non-perpendicular adjacent faces, some vertexes supposed to folded to the same position are close to each other after the initialization step. 
For each pair of vertexes $\mathbf{v}_a$, $\mathbf{v}_b$, if $|\mathbf{v}_a-\mathbf{v}_b|<\epsilon$, and one edge connecting to $\mathbf{v}_a$ has the same length with one edge connecting to $\mathbf{v}_b$.
\cxj{Add discussion about more than two vertexes. Improve figure by highlighting the merging edge and vertex. }

%Consider the initialization results can represent the ideal model partly, the disadjacent vertices that have one edge with same length can be regarded as targets that need to be located in the same place if their Euclidean distance is below a certain threshold.

\noindent
\textbf{symmetry detection} On account of the simplicity of the carton shape, the vertexes that have same length set of incident edges can be regarded as symmetric pair. While users select vertices that need to be merged together, symmetry detection can reduce repetitive work by giving users suggestions of symmetric vertices.
\cxj{Add a figure to illustrate this. }



\cxj{Once the user selects a suggested shape refinement operation, we optimize the 3D shape based on a series of geometric constraints.}
%We now introduce the constrains used in our construction method:
Given the current mesh whose vertex positions are defined as $\{\vo_i\}^{N}_{i=1}$, a new mesh with the same topology but new vertex positions $\{\vn_i\}^{N}_{i=1}$ is constructed.
%
The geometric constraints can be classified into two groups, face rigidity constraints and user-selected constraints. 
First, to keep the face rigidity, the constraints including edge length constraint, coplanarity, 


\noindent
\textbf{Edge length constraint} 
For each edge $\{e_j\}_{j=1...M}$, its length should be preserved when refine the carton shape.
Hence, its two endpoints $\mathbf{v}_{js}$ and end point $\mathbf{v}_{jt}$ should satisfy 
\begin{equation}
||\mathbf{v}_{js} - \mathbf{v}_{jt}||^2 = ||\mathbf{\hat{v}}_{js} - \mathbf{\hat{v}}_{jt}||^2.
\label{equ:edge}
\end{equation}
%to ensure that the length of each edge stays the same.

\noindent
\textbf{Coplanarity} For each face $\{p_k\}_{k=1 \dots P}$ and its normal $\mathbf{n}_k$, each line connected by two points $\mathbf{v}_{ka}, \mathbf{v}_{kb}$ on the plane is perpendicular to the normal. 
\cxj{where do you get the face nornal in the refined shape?}
\begin{equation}
\mathbf{n}_k \cdot (\mathbf{v}_{ka} - \mathbf{v}_{kb}) = 0.
\label{equ:coplane}
\end{equation}

\noindent
\textbf{Face rigidity} For each face $\{p_k\}_{k=1 \dots P}$, the length of each line connecting two non-adjacent points $\mathbf{v}_{ka}, \mathbf{v}_{kb}$ on the plane remains the same, so that the shape of each plane keeps unchanged.
\begin{equation}
||\mathbf{v}_{ka} - \mathbf{v}_{kb}||^2 = ||\hat{\mathbf{v}}_{ka} - \hat{\mathbf{v}}_{kb}||^2.
\label{equ:plane}
\end{equation}

The information acquired from user interaction will add enough constrains to solve the optimization problem, and one of the interaction is to choose the right given suggestion including points needed to be merged together. As for the point information, the constrains can be written like:

\noindent
\textbf{Merging vertexes} For any two vertexes $\mathbf{v}_p$ and $\mathbf{v}_q$ that are selected to be merged as the same vertex, we have 
\begin{equation}
\mathbf{v}_p - \mathbf{v}_q = 0.
\label{equ:point}
\end{equation}
%if these two points $\mathbf{v}_p$, $\mathbf{v}_q$ need to be moved into the same place.
\cxj{shape symmetry?}


When the above constrains still lead to an ill-posed problem, soft constraints will be added to keep the original positions of irrelevant vertexes. 


\noindent
\textbf{Irrelevant vertexes} $\{\mathbf{v}_i\}$ which are not in the same plane with $\mathbf{v}_p$ or $\mathbf{v}_q$, should be stay at their original location. 
We add these soft constraints by adding a small weight $w$, which is set 0.001 in our experiments. 
\begin{equation}
\mathbf{v}_i - \mathbf{\hat{v}}_i = 0.
\label{equ:irrelevant}
\end{equation}

Figure~\ref{fig:constrain} illustrates the necessity of basic three constrains.
\cxj{More detailed explaination on the figure. Explain the three constraints in the figure.}

\begin{figure}
	\centering
	\includegraphics[width=0.9\textwidth]{images/constrain.jpg}
	\caption{Given an initial state~(a), by choosing three points marked in red~(b) that need to be relocated together, basic three shape constrains can lead to the optimization result~(c). The bottom three are the results optimized lack of edge constrain, coplane constrain and plane constrain separately, compared to~(c), the results can not keep the basic shape well.}
	\label{fig:constrain}
\end{figure}

