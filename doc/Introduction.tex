\section{Introduction}
Cartons have been widely used in packaging industry to deliver various commodities including food items, daily necessities and electronic components. Instead of very basic packaging shapes like cubes, there exist multiple fantastic cartons to package wedding candies or take away coffee cups. These various designs increase much popularity, not to mention they're environmental friendly due to their recycling and degradability~\cite{Mullineux:2010:CSC:1739328.1739673}.

Cartons are usually designed based on experience and trial-and-error, though recent years there are softwares developed to help designers improve efficiency and productivity, for example, KASEMAKE~\cite{KASEMAKE} can help users pick the existing structural designs and feed in required basic size and material, also users can re-import the finished artwork to show the print on the structural designs and fold it into a three-dimensional view in seconds. There still consume much work and time to construct 3D model directly from 2D design layout. Moreover, it's sometimes intractable to fold an irregular box layout to final carton without instructions.

Researchers have studied papercraft problem for more than twenty years, they're primarily concerned with the following: (1) computational algorithm and mathematic analysis for folding origami, as in~\cite{Ida:2007:MOC:1802954.1803021,Lang:1996:CAO:237218.237249,xl-idetc-14}; (2) problems on folding polygon and unfolding polyhedron~\cite{Bern:2003:UPC:636968.636970,O'Rourke:1998:FUC:646319.686376,Rourke2008Unfolding}, and some prooves to show the problem complexity of folding to polyhedron~\cite{Biedl:2005:NFP:1090462.1646553,Biedl2004When,Lubiw1996When}; (3) other specific structure folding problems like Kinetogami which comprises multi-primitive and reconfigurable units folded from a single sheet of paper~\cite{Gao2013Kinetogami} and applications to show the folding behaviour of paper~\cite{Thiel1998,Kishi:1998:OFP:786112.786279,Nimnual2007Virtual,Shimanuki2009Construction}. Although previous works solved plenty problems in paper folding, they have not considered intelligent construction 3D models given 2D paper layouts. In this paper, we will focus on the carton folding problem and introduce shape optimization method to solve it.

In this paper, shape constrains for carton are proposed to optimize 2D design layout into the corresponding 3D realization. Moreover, an interactive design and exploration framework is developed to allow users visualize the structural layout in a 3D view, which can improve the efficiency a lot when designing a complicated paper package. The contribution of this paper includes the following:

(1) An interactive and exploration system is presented to manipulate the shape of 3D carton. It creates the corresponding flat polymesh by initialization and user interaction, and the stereo model of the carton is finally constructed.

(2) We propose shape constrains represented by a set of points to implement shape optimization. Observed from existing cartons,  constrains are summarized including edge constrain, coplane constrain and plane constrain to keep the shape of cartons. Besides, computer-aid detection like symmetry detection is brought to give users suggestions. 